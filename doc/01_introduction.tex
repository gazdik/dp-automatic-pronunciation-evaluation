\chapter{Úvod}

Vplyvom globalizácie sa dnes učí cudzí jazyk výrazne viac ľudí ako kedykoľvek predtým. Dôležitou, ale často podceňovanou, súčasťou cudzieho jazyka je  výslovnosť, ktorá je mnohokrát kľúčová pre správne dorozumenie. Efektívna výuka výslovnosti sa ale väčšinou nezaobíde bez individuálneho prístupu, čo si však väčšina študentov zväčša nemôže dovoliť. Z~tohto dôvodu sa automatické hodnotenie výslovnosti javí ako vhodná alternatíva. 

V~tejto práci sa budeme zaoberať automatickým hodnotením segmentálnych chýb, ktoré spočívajú vo vkladaní, vypúšťaní alebo zámene foném. K~tomuto účelu sa dnes výhradne využívajú systémy založené na rozpoznávaní reči, preto ani táto práca nebude výnimkou. Problém hodnotenia výslovnosti môžeme vnímať na dvoch úrovniach. Prvá z~nich spočíva v~detekcii nesprávnej výslovnosti, zatiaľ čo druhá sa snaží o~presnejšiu diagnostiku vzniknutej chyby, napríklad aká fonéma bola v~skutočnosti vyslovená. 

Zameriame sa výlučne na detekciu nesprávnej výslovnosti, nakoľko jej správne fungovanie priamo ovplyvňuje prípadnú diagnostiku. Našim cieľom bude primárne snaha o~zlepšenie existujúcich metód, ktoré sú k~tomuto účelu používané. Pokúsime sa o~to zavedením rôznych heuristík a na záver otestujeme vplyv použitia systémov rozpoznávania reči trénovaných na viacerých jazykoch.

Práca je štrukturovaná do niekoľkých kapitol. V~kapitole \ref{cha:asr-systems} sú všeobecne popísané systémy rozpoznávania reči. Nasleduje kapitola \ref{cha:neural-networks}, ktorá poskytuje základné informácie o~neurónových sieťach a ich použití k~rozpoznávaniu reči. V~kapitole \ref{cha:mispronunciation-evaluation} sú rozobraté dôležité prístupy používané na detekciu nesprávnej výslovnosti. Kapitola \ref{cha:datasets} sa venuje charakteristike datasetov, ktoré budú použité pri našich experimentoch. Asi najdôležitejšou časťou práce je kapitola \ref{cha:proposed-system} zameraná na návrh systému a popis zmien zacielených na zlepšenie úspešnosti. Na koniec popíšeme priebeh experimentov a analýzu dosiahnutých výsledkov v~kapitolách \ref{cha:experiments} a \ref{cha:evaluation}. 

% \begin{note}
%     \begin{itemize}
%         \item Efektívna výuka angličtiny sa väčšinou nezaobíde bez individuálnej výuky, čo si väčšina študentov často nemôže dovoliť. 
%     \end{itemize}
% \end{note}
