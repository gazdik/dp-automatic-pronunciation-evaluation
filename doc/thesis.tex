%==============================================================================
% tento soubor pouzijte jako zaklad
% this file should be used as a base for the thesis
% Autoři / Authors: 2008 Michal Bidlo, 2018 Jaroslav Dytrych
% Kontakt pro dotazy a připomínky: dytrych@fit.vutbr.cz
% Contact for questions and comments: dytrych@fit.vutbr.cz
%==============================================================================
% kodovani: UTF-8 (zmena prikazem iconv, recode nebo cstocs)
% encoding: UTF-8 (you can change it by command iconv, recode or cstocs)
%------------------------------------------------------------------------------
% zpracování / processing: make, make pdf, make clean
%==============================================================================
% Soubory, které je nutné upravit: / Files which have to be edited:
%   projekt-20-literatura-bibliography.bib - literatura / bibliography
%   projekt-01-kapitoly-chapters.tex - obsah práce / the thesis content
%   projekt-30-prilohy-appendices.tex - přílohy / appendices
%==============================================================================
% \documentclass[slovak]{fitthesis} % bez zadání - pro začátek práce, aby nebyl problém s překladem
%\documentclass[english]{fitthesis} % without assignment - for the work start to avoid compilation problem
\documentclass[slovak,zadani]{fitthesis} % odevzdani do wisu a/nebo tisk s barevnými odkazy - odkazy jsou barevné
% \documentclass[english,zadani]{fitthesis} % for submission to the IS FIT and/or print with color links - links are color
%\documentclass[zadani,print]{fitthesis} % pro černobílý tisk - odkazy jsou černé
%\documentclass[english,zadani,print]{fitthesis} % for the black and white print - links are black
%\documentclass[zadani,cprint]{fitthesis} % pro barevný tisk - odkazy jsou černé, znak VUT barevný
%\documentclass[english,zadani,cprint]{fitthesis} % for the print - links are black, logo is color
% * Je-li práce psaná v anglickém jazyce, je zapotřebí u třídy použít 
%   parametr english následovně:
%   If thesis is written in english, it is necessary to use 
%   parameter english as follows:
%      \documentclass[english]{fitthesis}
% * Je-li práce psaná ve slovenském jazyce, je zapotřebí u třídy použít 
%   parametr slovak následovně:
%   If the work is written in the Slovak language, it is necessary 
%   to use parameter slovak as follows:
%      \documentclass[slovak]{fitthesis}
% * Je-li práce psaná v anglickém jazyce se slovenským abstraktem apod., 
%   je zapotřebí u třídy použít parametry english a enslovak následovně:
%   If the work is written in English with the Slovak abstract, etc., 
%   it is necessary to use parameters english and enslovak as follows:
%      \documentclass[english,enslovak]{fitthesis}

% Základní balíčky jsou dole v souboru šablony fitthesis.cls
% Basic packages are at the bottom of template file fitthesis.cls
% zde můžeme vložit vlastní balíčky / you can place own packages here

% Kompilace po částech (rychlejší, ale v náhledu nemusí být vše aktuální)
% Compilation piecewise (faster, but not all parts in preview will be up-to-date)
\usepackage{subfiles}
\usepackage{standalone}

% Nastavení cesty k obrázkům
% Setting of a path to the pictures
%\graphicspath{{figures/}{./figures/}}
%\graphicspath{{figures/}{../figures/}}

%---rm---------------
\renewcommand{\rmdefault}{lmr}%zavede Latin Modern Roman jako rm / set Latin Modern Roman as rm
%---sf---------------
\renewcommand{\sfdefault}{qhv}%zavede TeX Gyre Heros jako sf
%---tt------------
\renewcommand{\ttdefault}{lmtt}% zavede Latin Modern tt jako tt

% vypne funkci šablony, která automaticky nahrazuje uvozovky,
% aby nebyly prováděny nevhodné náhrady v popisech API apod.
% disables function of the template which replaces quotation marks
% to avoid unnecessary replacements in the API descriptions etc.
\csdoublequotesoff

% =======================================================================
% balíček "hyperref" vytváří klikací odkazy v pdf, pokud tedy použijeme pdflatex
% problém je, že balíček hyperref musí být uveden jako poslední, takže nemůže
% být v šabloně
% "hyperref" package create clickable links in pdf if you are using pdflatex.
% Problem is that this package have to be introduced as the last one so it 
% can not be placed in the template file.
\ifWis
\ifx\pdfoutput\undefined % nejedeme pod pdflatexem / we are not using pdflatex
\else
  \usepackage{color}
  \usepackage[unicode,colorlinks,hyperindex,plainpages=false,pdftex]{hyperref}
  \definecolor{hrcolor-ref}{RGB}{223,52,30}
  \definecolor{hrcolor-cite}{HTML}{2F8F00}
  \definecolor{hrcolor-urls}{HTML}{092EAB}
  \hypersetup{
	linkcolor=hrcolor-ref,
	citecolor=hrcolor-cite,
	filecolor=magenta,
	urlcolor=hrcolor-urls
  }
  \def\pdfBorderAttrs{/Border [0 0 0] }  % bez okrajů kolem odkazů / without margins around links
  \pdfcompresslevel=9
\fi
\else % pro tisk budou odkazy, na které se dá klikat, černé / for the print clickable links will be black
\ifx\pdfoutput\undefined % nejedeme pod pdflatexem / we are not using pdflatex
\else
  \usepackage{xcolor}
  \usepackage[unicode,colorlinks,hyperindex,plainpages=false,pdftex,urlcolor=black,linkcolor=black,citecolor=black]{hyperref}
  \definecolor{links}{rgb}{0,0,0}
  \definecolor{anchors}{rgb}{0,0,0}
  \def\AnchorColor{anchors}
  \def\LinkColor{links}
  \def\pdfBorderAttrs{/Border [0 0 0] } % bez okrajů kolem odkazů / without margins around links
  \pdfcompresslevel=9
\fi
\fi
% Řešení problému, kdy klikací odkazy na obrázky vedou za obrázek
% This solves the problems with links which leads after the picture
\usepackage[all]{hypcap}

% Moje balíčky
\usepackage{tipa}
\usepackage{cite}
\usepackage{bm}
\usepackage{mathrsfs}
\usepackage{amsmath}
% limits underneath
\usepackage{mathtools}
\usepackage{subcaption}
\usepackage{framed}
\usepackage{xcolor} %frame color
\usepackage{pgfplots}
\usepackage{tipa}
\usepackage{textcomp}
\usepackage{tabularx}
\usepackage{tikz}

\usetikzlibrary{shapes.geometric, arrows, positioning, arrows.meta, calc}

\DeclareMathOperator*{\argmax}{arg\,max}
\newenvironment{note}{
    \def\FrameCommand{{\color{yellow}\vrule width 2pt}\hspace{2pt}}
    \MakeFramed{\advance\hsize-\width}
    \vspace{2pt}\noindent\hspace{-7pt}\vspace{3pt}
    }{\vspace{3pt}\endMakeFramed}

% Informace o práci/projektu / Information about the thesis
%---------------------------------------------------------------------------
\projectinfo{
  %Prace / Thesis
  project={DP},            %typ práce BP/SP/DP/DR  / thesis type (SP = term project)
  year={2019},             % rok odevzdání / year of submission
  date=\today,             % datum odevzdání / submission date
  %Nazev prace / thesis title
  title.cs={Automatické hodnocení anglické výslovnosti nerodilých mluvčích},  % název práce v češtině či slovenštině (dle zadání) / thesis title in czech language (according to assignment)
  title.en={Automatic Pronunciation Evaluation of Non-Native English Speakers}, % název práce v angličtině / thesis title in english
  %title.length={14.5cm}, % nastavení délky bloku s titulkem pro úpravu zalomení řádku (lze definovat zde nebo níže) / setting the length of a block with a thesis title for adjusting a line break (can be defined here or below)
  %Autor / Author
  author.name={Peter},   % jméno autora / author name
  author.surname={Gazdík},   % příjmení autora / author surname 
  %author.title.p={Bc.}, % titul před jménem (nepovinné) / title before the name (optional)
  %author.title.a={Ph.D.}, % titul za jménem (nepovinné) / title after the name (optional)
  %Ustav / Department
  department={UPGM}, % doplňte příslušnou zkratku dle ústavu na zadání: UPSY/UIFS/UITS/UPGM / fill in appropriate abbreviation of the department according to assignment: UPSY/UIFS/UITS/UPGM
  % Školitel / supervisor
  supervisor.name={Kateřina},   % jméno školitele / supervisor name 
  supervisor.surname={Žmolíková},   % příjmení školitele / supervisor surname
  supervisor.title.p={Ing.},   %titul před jménem (nepovinné) / title before the name (optional)
  %supervisor.title.a={Ph.D.},    %titul za jménem (nepovinné) / title after the name (optional)
  % Klíčová slova / keywords
  keywords.cs={automatické hodnotenie výslovnosti, výuka výslovnosti s~využitím počítača, automatické rozpoznávanie reči, hlboké neurónové siete, rekurentné neurónové siete}, % klíčová slova v českém či slovenském jazyce / keywords in czech or slovak language
  keywords.en={automatic pronunciation evaluation, computer-aided pronunciation training, automatic speech recognition, deep neural networks, recurrent neural networks}, % klíčová slova v anglickém jazyce / keywords in english
  %keywords.en={Here, individual keywords separated by commas will be written in English.},
  % Abstrakt / Abstract
  abstract.cs={Výuka anglickej výslovnosti s využitím počítača sa v súčasnej dobe stáva čoraz viac populárnejšou. Napriek tomu presnosť týchto systémov je stále pomerne nízka. Táto diplomová práca sa preto zameriava na zlepšenie existujúcich metód automatického hodnotenia výslovnosti. V prvej časti práce je uvedený prehľad v súčasnosti používaných techník v tejto oblasti. Následne bol navrhnutý systém využívajúci dva rôzne prístupy. Dosiahnuté výsledky ukazujú znateľné zlepšenie oproti referenčnému systému.}, % abstrakt v českém či slovenském jazyce / abstract in czech or slovak language
  abstract.en={Computer-Assisted Pronunciation Training (CAPT) is becoming more and more popular these days. However, the accuracy of existing CAPT systems is still quite low. Therefore, this diploma thesis focuses on improving existing methods for automatic pronunciation evaluation on the segmental level. The first part describes common techniques for this task. Afterwards, we proposed the system based on two approaches. Finally, performed experiments show significant improvement over the reference system.}, % abstrakt v anglickém jazyce / abstract in english
  %abstract.fen={An abstract of the work in English will be written in this paragraph.},
  % Prohlášení (u anglicky psané práce anglicky, u slovensky psané práce slovensky) / Declaration (for thesis in english should be in english)
  declaration={Prehlasujem, že som túto diplomovú prácu vypracoval samostatne pod vedením Ing.\,Kateřiny Žmolíkové a uviedol som všetky literárne pramene a publikácie, z~ktorých som čerpal.},
  %declaration={Hereby I declare that this bachelor's thesis was prepared as an original author’s work under the supervision of Mr. X
% The supplementary information was provided by Mr. Y
% All the relevant information sources, which were used during preparation of this thesis, are properly cited and included in the list of references.},
  % Poděkování (nepovinné, nejlépe v jazyce práce) / Acknowledgement (optional, ideally in the language of the thesis)
 acknowledgment={Rád by som poďakoval mojej vedúcej práce Katke Žmolíkovej za cenné pripomienky a~množstvo času, ktorý mi venovala pri konzultáciách.},
  %acknowledgment={Here it is possible to express thanks to the supervisor and to the people which provided professional help
%(external submitter, consultant, etc.).},
  % Rozšířený abstrakt (cca 3 normostrany) - lze definovat zde nebo níže / Extended abstract (approximately 3 standard pages) - can be defined here or below
  %extendedabstract={Do tohoto odstavce bude zapsán rozšířený výtah (abstrakt) práce v českém (slovenském) jazyce.},
  %faculty={FIT}, % FIT/FEKT/FSI/FA/FCH/FP/FAST/FAVU/USI/DEF
  faculty.cs={Fakulta informačních technologií}, % Fakulta v češtině - pro využití této položky výše zvolte fakultu DEF / Faculty in Czech - for use of this entry select DEF above
  faculty.en={Faculty of Information Technology}, % Fakulta v angličtině - pro využití této položky výše zvolte fakultu DEF / Faculty in English - for use of this entry select DEF above
  %department.cs={Ústav matematiky}, % Ústav v češtině - pro využití této položky výše zvolte ústav DEF nebo jej zakomentujte / Department in Czech - for use of this entry select DEF above or comment it out
  %department.en={Institute of Mathematics} % Ústav v angličtině - pro využití této položky výše zvolte ústav DEF nebo jej zakomentujte / Department in English - for use of this entry select DEF above or comment it out
}

% Rozšířený abstrakt (cca 3 normostrany) - lze definovat zde nebo výše / Extended abstract (approximately 3 standard pages) - can be defined here or above
%\extendedabstract{Do tohoto odstavce bude zapsán výtah (abstrakt) práce v českém (slovenském) jazyce.}

% nastavení délky bloku s titulkem pro úpravu zalomení řádku - lze definovat zde nebo výše / setting the length of a block with a thesis title for adjusting a line break - can be defined here or above
%\titlelength{14.5cm}


% řeší první/poslední řádek odstavce na předchozí/následující stránce
% solves first/last row of the paragraph on the previous/next page
\clubpenalty=10000
\widowpenalty=10000

% checklist
\newlist{checklist}{itemize}{1}
\setlist[checklist]{label=$\square$}

\begin{document}
  % Vysazeni titulnich stran / Typesetting of the title pages
  % ----------------------------------------------
  \maketitle
  % Obsah
  % ----------------------------------------------
  \setlength{\parskip}{0pt}

  {\hypersetup{hidelinks}\tableofcontents}
  
  % Seznam obrazku a tabulek (pokud prace obsahuje velke mnozstvi obrazku, tak se to hodi)
  % List of figures and list of tables (if the thesis contains a lot of pictures, it is good)
  \ifczech
    \renewcommand\listfigurename{Seznam obrázků}
  \fi
  \ifslovak
    \renewcommand\listfigurename{Zoznam obrázkov}
  \fi
  % \listoffigures
  
  \ifczech
    \renewcommand\listtablename{Seznam tabulek}
  \fi
  \ifslovak
    \renewcommand\listtablename{Zoznam tabuliek}
  \fi
  % \listoftables 

  \ifODSAZ
    \setlength{\parskip}{0.5\bigskipamount}
  \else
    \setlength{\parskip}{0pt}
  \fi

  % vynechani stranky v oboustrannem rezimu
  % Skip the page in the two-sided mode
  \iftwoside
    \cleardoublepage
  \fi

  % Text prace / Thesis text
  % ----------------------------------------------
  %\input{01_chapters}
  
  % Kompilace po částech (viz výše, nutno odkomentovat)
  % Compilation piecewise (see above, it is necessary to uncomment it)
  \subfile{01_introduction}
  \subfile{02_speech_recognition}
  \subfile{03_neural_networks}
  \subfile{04_mispronunciation_evaluation} 
  \subfile{05_datasets}  
  \subfile{06_proposed_system}  
  \subfile{07_experiments}  
  \subfile{08_evaluation}
  \subfile{09_conclusion} 

  % Pouzita literatura / Bibliography
  % ----------------------------------------------
\ifslovak
  \makeatletter
  \def\@openbib@code{\addcontentsline{toc}{chapter}{Literatúra}}
  \makeatother
  \bibliographystyle{bib-styles/slovakiso}
\else
  \ifczech
    \makeatletter
    \def\@openbib@code{\addcontentsline{toc}{chapter}{Literatura}}
    \makeatother
    \bibliographystyle{bib-styles/czechiso}
  \else 
    \makeatletter
    \def\@openbib@code{\addcontentsline{toc}{chapter}{Bibliography}}
    \makeatother
    \bibliographystyle{bib-styles/englishiso}
  %  \bibliographystyle{alpha}
  \fi
\fi
  \begin{flushleft}
  \bibliography{10_bibliography.bib}
  \end{flushleft}

  % vynechani stranky v oboustrannem rezimu
  % Skip the page in the two-sided mode
  \iftwoside
    \cleardoublepage
  \fi

  % Prilohy / Appendices
  % ---------------------------------------------
  \appendix
\ifczech
  \renewcommand{\appendixpagename}{Přílohy}
  \renewcommand{\appendixtocname}{Přílohy}
  \renewcommand{\appendixname}{Příloha}
\fi
\ifslovak
  \renewcommand{\appendixpagename}{Prílohy}
  \renewcommand{\appendixtocname}{Prílohy}
  \renewcommand{\appendixname}{Príloha}
\fi
%  \appendixpage

% vynechani stranky v oboustrannem rezimu
% Skip the page in the two-sided mode
%\iftwoside
%  \cleardoublepage
%\fi
  
\ifslovak
%  \section*{Zoznam príloh}
%  \addcontentsline{toc}{section}{Zoznam príloh}
\else
  \ifczech
%    \section*{Seznam příloh}
%    \addcontentsline{toc}{section}{Seznam příloh}
  \else
%    \section*{List of Appendices}
%    \addcontentsline{toc}{section}{List of Appendices}
  \fi
\fi
  \startcontents[chapters]
  \setlength{\parskip}{0pt}
  % seznam příloh / list of appendices
  % \printcontents[chapters]{l}{0}{\setcounter{tocdepth}{2}}
  
  \ifODSAZ
    \setlength{\parskip}{0.5\bigskipamount}
  \else
    \setlength{\parskip}{0pt}
  \fi
  
  % vynechani stranky v oboustrannem rezimu
  \iftwoside
    \cleardoublepage
  \fi
  
  % Přílohy / Appendices
  %------------------------------------------------------------------
\chapter{Obsah priloženého pamäťového média} \label{appendix:cd}
%------------------------------------------------------------------

Na priloženom pamäťovom médiu sa nachádza

\begin{itemize}
  \item elektronická verzia tohoto dokumentu spolu zo zdrojovými súbormi v jazyku \LaTeX,
  \item archív obsahujúci skripty a zdrojové kódy, ktoré realizujú popísané experimenty.
\end{itemize}

\noindent Obsah bližšie popisuje súbor \texttt{README.txt}, ktorý sa nachádza v koreňovom adresári.


% Umístění obsahu paměťového média do příloh je vhodné konzultovat s vedoucím
% Placing of table of contents of the memory media here should be consulted with a supervisor

% \chapter{Rozdelenie ISLE datasetu} \label{appendix:isle-train-test-sets}

% % Please add the following required packages to your document preamble:
% % \usepackage{booktabs}
% \begin{table}[]
% \centering
% \begin{tabular}{@{}l|lllll@{}}
% \toprule
% Testovacia sada & \multicolumn{5}{l}{Trénovacia sada}                  \\ \midrule
% SESS0006        & SESS0012 & SESS0183 & SESS0191 & SESS0126 & SESS0134 \\
% SESS0011        & SESS0021 & SESS0184 & SESS0192 & SESS0127 & SESS0135 \\
% SESS0015        & SESS0161 & SESS0185 & SESS0193 & SESS0128 & SESS0136 \\
% SESS0020        & SESS0162 & SESS0186 & SESS0003 & SESS0129 & SESS0137 \\
% SESS0041        & SESS0163 & SESS0187 & SESS0040 & SESS0130 & SESS0138 \\
% SESS0121        & SESS0164 & SESS0188 & SESS0123 & SESS0131 & SESS0140 \\
% SESS0122        & SESS0181 & SESS0189 & SESS0124 & SESS0132 &          \\
% SESS0139        & SESS0182 & SESS0190 & SESS0125 & SESS0133 &          \\ \bottomrule
% \end{tabular}
% \label{tab:isle-train-test-sets}
% \caption{Rozdelenie ISLE datasetu na testovaciu a trénovaciu sadu, pričom údaje v tabuľke predstavujú identifikátory jednotlivých rečníkov.}
% \end{table}

% \chapter{Obsah priloženého pamäťového média}

%\chapter{Manuál}

%\chapter{Konfiguračný súbor} % Configuration file

%\chapter{Plagát} % poster

  
  % Kompilace po částech (viz výše, nutno odkomentovat)
  % Compilation piecewise (see above, it is necessary to uncomment it)
  %\subfile{projekt-30-prilohy-appendices}

\end{document}
