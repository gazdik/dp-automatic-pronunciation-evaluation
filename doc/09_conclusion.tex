
\chapter{Záver} \label{cha:conclusion}

V~tejto diplomovej práci sme sa venovali automatickému hodnoteniu výslovnosti na segmentálnej úrovni a to najmä z~pohľadu detekcie chýb. Bližšie sme sa zamerali na dva prístupy hodnotenia výslovnosti, konkrétne na metódy založené na aposteriórnej pravdepodobnosti foném a metódy priamej klasifikácie pomocou neurónových sietí.

Na základe vykonaných experimentov sme ukázali, že nami navrhnutý systém dosahuje nad použitým nenatívnym datasetom výrazne lepšie výsledky ako referenčná práca, z~ktorej sme vychádzali. Pri vybraných metódach sme sa ďalej pokúšali o~zlepšenie navrhnutého systému multilingválnym trénovaním akustického modelu na niekoľkých rôznych jazykoch. Tento prístup priniesol taktiež významné zlepšenie, avšak výsledná chyba sa nám javila stále pomerne vysoká.

Preto sme na záver práce vykonali bližšiu analýzu výsledkov u~vybranej metódy. Týmto postupom sme zistili, že výsledky automatického hodnotenia sa výrazne líšia v~závislosti na hodnotenej fonéme. To nás viedlo aj na analýzu použitého nenatívneho datasetu, kde sme na jeho podčasti skúmali odlišnosti medzi prepismi pochádzajúcimi od rôznych anotátorov. Takto sme dospeli k~záveru, že dataset trpí v~určitých prípadoch významnou nekonzistenciou, ktorá má nezanedbateľný dopad na dosiahnuté výsledky. Zároveň sme však pri jednej z~foném pozorovali vysokú chybu pri automatickýck hodnoteniach napriek tomu, že anotácie boli pri nej dostatočne konzistenté. Dá sa teda usudzovať, že v~tejto oblasti je priestor na ďalšie zlepšenie.

V~rámci ďalšej práce preto navrhujeme bližšie preskúmať charakter jednotlivých chýb v~nenatívnom datasete. Okrem toho by bolo taktiež vhodné overiť dosiahnuté výsledky aj na nejakom inom nenatívnom datasete s~vyššou konzistenciou anotácii. 

